\documentclass[xcolor=pdflatex,dvipsnames,table]{beamer}
\usepackage{epsfig,graphicx}
\usepackage{palatino}
\usepackage{fancybox}
\usepackage{relsize}
\usepackage[procnames]{listings}
\usepackage{hyperref}


% fatter TT font
\renewcommand*\ttdefault{txtt}
% another TT, suggested by Alex
% \usepackage{inconsolata}
% \usepackage[T1]{fontenc} % needed as well?

\usepackage[procnames]{listings}

% shared in slides and book

\lstdefinelanguage{chisel}{
  morekeywords={abstract,case,catch,class,def,%
    do,else,extends,false,final,finally,%
    for,if,implicit,import,match,mixin,%
    new,null,object,override,package,%
    private,protected,requires,return,sealed,%
    super,this,throw,trait,true,try,%
    type,val,var,while,with,yield},
  otherkeywords={=>,<-,<\%,<:,>:,\#,@},
  sensitive=true,
  morecomment=[l]{//},
  morecomment=[n]{/*}{*/},
  morestring=[b]",
  morestring=[b]',
  morestring=[b]"""
}

\usepackage{color}
\definecolor{dkgreen}{rgb}{0,0.6,0}
\definecolor{gray}{rgb}{0.5,0.5,0.5}
\definecolor{mauve}{rgb}{0.58,0,0.82}

% Default settings for code listings
\ifbook
\lstset{%frame=lines,
  language=chisel,
  aboveskip=3mm,
  belowskip=3mm,
  showstringspaces=false,
  columns=fixed, % basewidth=\mybasewidth,
  basicstyle={\small\ttfamily},
  numbers=none,
  numberstyle=\footnotesize,
  % identifierstyle=\color{red},
  breaklines=true,
  breakatwhitespace=true,
  procnamekeys={def, val, var, class, trait, object, extends},
  % procnamestyle=\ttfamily,
  tabsize=2,
  float
}
\else
\lstset{%frame=lines,
  language=chisel,
  aboveskip=3mm,
  belowskip=3mm,
  showstringspaces=false,
  columns=fixed, % basewidth=\mybasewidth,
  basicstyle={\small\ttfamily},
  numbers=none,
  numberstyle=\footnotesize\color{gray},
  % identifierstyle=\color{red},
  keywordstyle=\color{blue},
  commentstyle=\color{dkgreen},
  stringstyle=\color{mauve},
  breaklines=true,
  breakatwhitespace=true,
  procnamekeys={def, val, var, class, trait, object, extends},
  procnamestyle=\ttfamily\color{red},
  tabsize=2,
  float
}
\fi

\lstnewenvironment{chisel}[1][]
{\lstset{language=chisel,#1}}
{}

\newcommand{\shortlist}[1]{{\lstinputlisting[nolol]{#1}}}

\newcommand{\longlist}[3]{{\lstinputlisting[float, caption={#2}, label={#3}, frame=tb, captionpos=b]{#1}}}

\newcommand{\verylonglist}[3]{{\lstinputlisting[caption={#2}, label={#3}, frame=tb, captionpos=b]{#1}}}


\hypersetup{
  linkcolor  = black,
%  citecolor  = blue,
  urlcolor   = blue,
  colorlinks = true,
}


\newcommand{\todo}[1]{{\emph{TODO: #1}}}
\newcommand{\martin}[1]{{\color{blue} Martin: #1}}
\newcommand{\abcdef}[1]{{\color{red} Author2: #1}}

% uncomment following for final submission
%\renewcommand{\todo}[1]{}
%\renewcommand{\martin}[1]{}
%\renewcommand{\author2}[1]{}


\title{Digital Design in the 21st Century: Chisel}
\author{Martin Schoeberl}
\date{\today}
\institute{DTU Compute}

\begin{document}

\begin{frame}
\titlepage
\end{frame}

\begin{frame}[fragile]{Notes}
\begin{itemize}
\item a Mux for any data type
\item generation of logic in code, e.g. ROM or binary to BCD instead of Java program
\item Connectors
\item Tester (e.g. the counter)
\item xxx
\end{itemize}
\end{frame}


\begin{frame}[fragile]{Goals for this Intro}
\begin{itemize}
\item Get an idea what Chisel is
\begin{itemize}
\item Will show you code snippets
\end{itemize}
\item Reconsider how to describe hardware
\item Some experience report from usage at DTU
\item Pointers to more information
\end{itemize}
\end{frame}

\begin{frame}[fragile]{Talk abstract}

Date: Tu 19/04/2016, 11:00-12:00
Room: 123/322

Title: Hardware Design in the 21st Century: with the Object Oriented
and Functional Language Chisel

Chisel is a hardware construction language implemented as a
domain specific language in Scala. Therefore, the full power of
a modern programming language is available to describe hardware
and, more important, hardware generators. Chisel has been developed
at UC Berkeley and successfully used for several tape outs of RISC-V.
Here at DTU we used Chisel in the T-CREST project and in teaching
advanced computer architecture. Besides presenting small code
examples in Chisel I will report on experiences on using Chisel in
the t-CREST project and in teaching.

Martin Schoeberl
\end{frame}

\begin{frame}[fragile]{Chisel}
\begin{itemize}
\item A hardware \emph{construction} language
\item If it compiles it is synthesizable hardware 
\item Single source two targets:
\begin{itemize}
\item C based cycle accurate simulator
\item Verilog for synthesis
\end{itemize}
\item Embedded in Scala
\begin{itemize}
\item Full power of Scala available
\item But to start with, no Scala knowledge needed
\end{itemize}
\item Developed at UC Berkeley
\end{itemize}
\end{frame}





\begin{frame}[fragile]{A Register}
\begin{chisel}
val cntReg = Reg(init = UInt(0, 25))

cntReg := cntReg + UInt(1)
\end{chisel}
\begin{itemize}
\item Type inferred by initial value (reset value)
\item No need to specify a clock or reset signal
\end{itemize}
\begin{itemize}
\item Also definition with an input signal connected:
\end{itemize}
\begin{chisel}
val r = Reg(next = nextVal) 
\end{chisel}

\end{frame}

\begin{frame}[fragile]{Expressions}
\begin{chisel}
code
\end{chisel}
\begin{itemize}
\item Width inference
\item xxx
\item xxx
\end{itemize}
\end{frame}

\begin{frame}[fragile]{Bundles}
\begin{chisel}
code
\end{chisel}
\begin{itemize}
\item xxx
\item xxx
\item xxx
\end{itemize}
\end{frame}

\begin{frame}[fragile]{Modules}
\begin{chisel}
code
\end{chisel}
\begin{itemize}
\item xxx
\item xxx
\item xxx
\end{itemize}
\end{frame}

\begin{frame}[fragile]{Connections}
\begin{chisel}
code
\end{chisel}
\begin{itemize}
\item xxx
\item xxx
\item xxx
\end{itemize}
\end{frame}

%%%%%%%%%%%%%%%%%
\begin{frame}[fragile]{zzz}
\begin{chisel}
code
\end{chisel}
\begin{itemize}
\item xxx
\item xxx
\item xxx
\end{itemize}
\end{frame}

\begin{frame}[fragile]{yyy}
\begin{itemize}
\item xxx
\item xxx
\item xxx
\item xxx
\item xxx
\end{itemize}
\end{frame}
%%%%%%%%%%%%%%%%%

\begin{frame}[fragile]{Scala}
\begin{itemize}
\item Object oriented
\item Functional
\item Strongly typed
\begin{itemize}
\item With very good type interference
\end{itemize}
\item Could be seen as Java++
\item Compiled to the JVM
\item Good Java interoperability
\begin{itemize}
\item Many libraries available
\end{itemize}
\end{itemize}
\end{frame}


\begin{frame}[fragile]{Hello World in Chisel}
\begin{chisel}
class Hello extends Module {
  val io = new Bundle {
    val led = UInt(OUTPUT, 1)
  }
  val CNT_MAX = UInt(16000000 / 2 - 1);
  val r1 = Reg(init = UInt(0, 25))
  val blk = Reg(init = UInt(0, 1))

  r1 := r1 + UInt(1)
  when(r1 === CNT_MAX) {
    r1 := UInt(0)
    blk := ~blk
  }
  io.led := blk
}
\end{chisel}
\end{frame}

\begin{frame}[fragile]{Hello World in Chisel}
\begin{chisel}
git clone https://github.com/schoeberl/chisel-examples.git
\end{chisel}
\begin{itemize}
\item or download from \url{https://github.com/schoeberl/chisel-examples}
\item This contains a minimal Chisel project with the blinking LED
\item Has ready to use project files for:
\begin{itemize}
\item Altera DE0
\item Altera DE1
\item Altera DE2-115
\item BeMicro
\end{itemize}
\item Plus a simple ALU for HW test and showing Chisel Tester
\item Plus some more examples to explore
\end{itemize}
\end{frame}

\end{document}