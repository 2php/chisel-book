\documentclass[xcolor=pdflatex,dvipsnames,table]{beamer}
\usepackage{epsfig,graphicx}
\usepackage{palatino}
\usepackage{fancybox}
\usepackage{relsize}
\usepackage[procnames]{listings}
\usepackage{hyperref}


% fatter TT font
\renewcommand*\ttdefault{txtt}
% another TT, suggested by Alex
% \usepackage{inconsolata}
% \usepackage[T1]{fontenc} % needed as well?

\usepackage[procnames]{listings}

% shared in slides and book

\lstdefinelanguage{chisel}{
  morekeywords={abstract,case,catch,class,def,%
    do,else,extends,false,final,finally,%
    for,if,implicit,import,match,mixin,%
    new,null,object,override,package,%
    private,protected,requires,return,sealed,%
    super,this,throw,trait,true,try,%
    type,val,var,while,with,yield},
  otherkeywords={=>,<-,<\%,<:,>:,\#,@},
  sensitive=true,
  morecomment=[l]{//},
  morecomment=[n]{/*}{*/},
  morestring=[b]",
  morestring=[b]',
  morestring=[b]"""
}

\usepackage{color}
\definecolor{dkgreen}{rgb}{0,0.6,0}
\definecolor{gray}{rgb}{0.5,0.5,0.5}
\definecolor{mauve}{rgb}{0.58,0,0.82}

% Default settings for code listings
\ifbook
\lstset{%frame=lines,
  language=chisel,
  aboveskip=3mm,
  belowskip=3mm,
  showstringspaces=false,
  columns=fixed, % basewidth=\mybasewidth,
  basicstyle={\small\ttfamily},
  numbers=none,
  numberstyle=\footnotesize,
  % identifierstyle=\color{red},
  breaklines=true,
  breakatwhitespace=true,
  procnamekeys={def, val, var, class, trait, object, extends},
  % procnamestyle=\ttfamily,
  tabsize=2,
  float
}
\else
\lstset{%frame=lines,
  language=chisel,
  aboveskip=3mm,
  belowskip=3mm,
  showstringspaces=false,
  columns=fixed, % basewidth=\mybasewidth,
  basicstyle={\small\ttfamily},
  numbers=none,
  numberstyle=\footnotesize\color{gray},
  % identifierstyle=\color{red},
  keywordstyle=\color{blue},
  commentstyle=\color{dkgreen},
  stringstyle=\color{mauve},
  breaklines=true,
  breakatwhitespace=true,
  procnamekeys={def, val, var, class, trait, object, extends},
  procnamestyle=\ttfamily\color{red},
  tabsize=2,
  float
}
\fi

\lstnewenvironment{chisel}[1][]
{\lstset{language=chisel,#1}}
{}

\newcommand{\shortlist}[1]{{\lstinputlisting[nolol]{#1}}}

\newcommand{\longlist}[3]{{\lstinputlisting[float, caption={#2}, label={#3}, frame=tb, captionpos=b]{#1}}}

\newcommand{\verylonglist}[3]{{\lstinputlisting[caption={#2}, label={#3}, frame=tb, captionpos=b]{#1}}}


\hypersetup{
  linkcolor  = black,
%  citecolor  = blue,
  urlcolor   = blue,
  colorlinks = true,
}

\newcommand{\code}[1]{{\texttt{#1}}}

\beamertemplatenavigationsymbolsempty
\setbeamertemplate{footline}[frame number]

\newcommand{\todo}[1]{{\emph{TODO: #1}}}
\newcommand{\martin}[1]{{\color{blue} Martin: #1}}
\newcommand{\abcdef}[1]{{\color{red} Author2: #1}}

% uncomment following for final submission
%\renewcommand{\todo}[1]{}
%\renewcommand{\martin}[1]{}
%\renewcommand{\author2}[1]{}


\title{Digital Design in the 21st Century: Chisel}
\author{Martin Schoeberl}
\date{\today}
\institute{Technical University of Denmark}

\begin{document}

\begin{frame}
\titlepage
\end{frame}


\begin{frame}[fragile]{Goals for this Intro}
\begin{itemize}
\item Get an idea what Chisel is
\begin{itemize}
\item Will show you code snippets
\end{itemize}
\item Reconsider how to describe hardware
\item Some experience report from usage at DTU
\item Pointers to more information
\item A first high level view of the main features of Chisel
\end{itemize}
\end{frame}

%\begin{frame}[fragile]{Talk abstract}
%
%Date: Tu 19/04/2016, 11:00-12:00
%Room: 123/322
%
%Title: Hardware Design in the 21st Century: with the Object Oriented
%and Functional Language Chisel
%
%Chisel is a hardware construction language implemented as a
%domain specific language in Scala. Therefore, the full power of
%a modern programming language is available to describe hardware
%and, more important, hardware generators. Chisel has been developed
%at UC Berkeley and successfully used for several tape outs of RISC-V.
%Here at DTU we used Chisel in the T-CREST project and in teaching
%advanced computer architecture. Besides presenting small code
%examples in Chisel I will report on experiences on using Chisel in
%the t-CREST project and in teaching.
%
%Martin Schoeberl
%\end{frame}

\begin{frame}[fragile]{Chisel}
\begin{itemize}
\item A hardware \emph{construction} language
\begin{itemize}
\item If it compiles it is synthesisable hardware 
\item Say goodby to your unintended latches
\end{itemize}
\item Single source two targets
\begin{itemize}
\item C based cycle accurate simulator
\item Verilog for synthesis
\end{itemize}
\item Embedded in Scala
\begin{itemize}
\item Full power of Scala available
\item But to start with, no Scala knowledge needed
\end{itemize}
\item Developed at UC Berkeley
\end{itemize}
\end{frame}

\begin{frame}[fragile]{Some Notes on Scala}
\begin{itemize}
\item Object oriented
\item Functional
\item Strongly typed
\begin{itemize}
\item With very good type inference
\end{itemize}
\item Could be seen as Java++
\item Compiled to the JVM
\item Good Java interoperability
\begin{itemize}
\item Many libraries available
\end{itemize}
\end{itemize}
\end{frame}

\begin{frame}[fragile]{Chisel vs. Scala}
\begin{itemize}
\item A Chisel hardware description is a Scala program
\item Chisel is a Scala library
\item When the program is executed it generates hardware
\item Chisel is a so-called \emph{embedded domain-specific language}
\end{itemize}
\end{frame}

\begin{frame}[fragile]{Expressions are Combinational Circuits}
\begin{chisel}
(a | b) & ~(c ^ d)

val addVal = a + b
val orVal = a | b
val boolVal = a >= b
\end{chisel}
\begin{itemize}
\item The usual operations 
\item Simple name assignment with val
\item Width inference
\item Type inference
\item Types: Bits, UInt, SInt, Bool
\end{itemize}
\end{frame}

\begin{frame}[fragile]{Registers}
\begin{chisel}
val cntReg = Reg(init = UInt(0, 25))

cntReg := cntReg + UInt(1)
\end{chisel}
\begin{itemize}
\item Type inferred by initial value (= reset value)
\item No need to specify a clock or reset signal
\end{itemize}
\begin{itemize}
\item Also definition with an input signal connected:
\end{itemize}
\begin{chisel}
val r = Reg(next = nextVal) 
\end{chisel}
\end{frame}

\begin{frame}[fragile]{Functional Abstraction}
\begin{chisel}
  def addSub(add: Bool, a: UInt, b: UInt) =
    Mux(add, a+b, a-b)

  val res = addSub(cond, a, b)
  
  def rising(d: Bool) = d && !Reg(next = d)
\end{chisel}
\begin{itemize}
\item Functions for repeated pieces of logic
\item May contain state
\item Functions may return \emph{hardware}
\end{itemize}
\end{frame}


\begin{frame}[fragile]{Bundles}
\begin{chisel}
class DecodeExecute extends Bundle {
  val rs1 = UInt(width = 32)
  val rs2 = UInt(width = 32)
  val immVal = UInt(width = 32)
  val aluOp = new AluOp()
}
\end{chisel}
\begin{itemize}
\item Collection of values in named fields 
\item Like struct or record
\end{itemize}
\end{frame}

\begin{frame}[fragile]{Vectors}
\begin{chisel}
val myVec = Vec.fill(3){ SInt(width = 10) }

myVec(0) := SInt(-3)
val y = myVec(2)
\end{chisel}
\begin{itemize}
\item Indexable vector of elements
\item Bundles and Vecs can be arbitrary nested
\end{itemize}
\end{frame}

\begin{frame}[fragile]{IO Ports}
\begin{chisel}
class Channel extends Bundle {
  val data = Bits(INPUT, 8)
  val ready = Bool(OUTPUT)
  val valid = Bool(INPUT)
}
\end{chisel}
\begin{itemize}
\item Ports are Bundles with directions
\item Direction can also be assigned at instantiation:
\end{itemize}
\begin{chisel}
class ExecuteIO extends Bundle {
  val dec = new DecodeExecute().asInput
  val mem = new ExecuteMemory().asOutput
}
\end{chisel}
\end{frame}

\begin{frame}[fragile]{Modules}
\begin{chisel}
class Adder extends Module {
  val io = new Bundle {
    val a = UInt(INPUT, 4)
    val b = UInt(INPUT, 4)
    val result = UInt(OUTPUT, 4)
  }

  val addVal = io.a + io.b
  io.result := addVal
}
\end{chisel}
\begin{itemize}
\item Organization of components
\item IO ports defined as a Bundle named \code{io}
\item Created (instantiated) with:
\end{itemize}
\begin{chisel}
val adder = Module(new Adder())
\end{chisel}
\end{frame}

\begin{frame}[fragile]{Hello World in Chisel}
\begin{chisel}
class Hello extends Module {
  val io = new Bundle {
    val led = UInt(OUTPUT, 1)
  }
  val CNT_MAX = UInt(20000000 / 2 - 1);
  
  val cntReg = Reg(init = UInt(0, 25))
  val blkReg = Reg(init = UInt(0, 1))

  cntReg := cntReg + UInt(1)
  when(cntReg === CNT_MAX) {
    cntReg := UInt(0)
    blkReg := ~blkReg
  }
  io.led := blkReg
}
\end{chisel}
\end{frame}

\begin{frame}[fragile]{Connections}
\begin{itemize}
\item Simple connections just with assignments, e.g.,
\begin{chisel}
  adder.io.a := ina
  adder.io.b := inb
\end{chisel}
\item Automatic bulk connections between components
\begin{chisel}
  dec.io <> exe.io
  mem.io <> exe.io
\end{chisel}
\end{itemize}
\end{frame}

\begin{frame}[fragile]{Generic Components}
\begin{chisel}
val c = Mux(cond, a, b)
\end{chisel}
\begin{itemize}
\item This is a multiplexer
\item Input can be any type
\end{itemize}
\end{frame}

\begin{frame}[fragile]{Testing}
\begin{chisel}
class CounterTester(c: Counter) extends Tester(c) {
  for (i <- 0 until 5) {
    println(peek(c.io.out))
    step(1)
  }
}
\end{chisel}
\begin{itemize}
\item Within Chisel with a tester (= Scala program)
\item May include waveform generation
\item poke and peek to set and read values
\item printf in simulation on rising edge
\begin{chisel}
printf("Counting %x\n", r1)
\end{chisel}
\end{itemize}
\end{frame}

\begin{frame}[fragile]{Component Generation}
\begin{chisel}
val cores = new Array[Module](32)

for (j <- 0 until 32)
  cores(j) = Module(new CPU())
\end{chisel}
\begin{itemize}
\item Use Scala array to collect components
\item Generation with a Scala loop
\end{itemize}
\end{frame}

\begin{frame}[fragile]{Conditional Component Generation}
\begin{chisel}
val icache =
  if (TYPE == METHOD)
    Module(new MCache())
  else if (TYPE == LINE)
    Module(new ICache())
  else
    ChiselError.error("Unsupported Type")
\end{chisel}
\begin{itemize}
\item Use Scala if/else for conditional component types
\item Code example from Patmos
\item We parse an XML file for the configuration
\end{itemize}
\end{frame}

\begin{frame}[fragile]{Logic Generation}
\begin{itemize}
\item Read a file into a table
\begin{itemize}
\item E.g., to read in ROM content for a processor
\end{itemize}
\item Generate a truth table algorithmically
\begin{itemize}
\item E.g., generate binary to BCD translation
\end{itemize}
\item Use the full power of Scala
\end{itemize}
\begin{chisel}
val source = fromFile(fileName)
val byteArray = source.map(_.toByte).toArray
source.close()
val arr = new Array[Bits](byteArray.length)
for (i <- 0 until byteArray.length) {
  arr(i) = Bits(byteArray(i), 8)
}
val rom = Vec[Bits](arr)
\end{chisel}
\end{frame}
%%%%%%%%%%%%%%%%%
%\begin{frame}[fragile]{zzz}
%\begin{chisel}
%code
%\end{chisel}
%\begin{itemize}
%\item xxx
%\item xxx
%\item xxx
%\end{itemize}
%\end{frame}
%
%\begin{frame}[fragile]{yyy}
%\begin{itemize}
%\item xxx
%\item xxx
%\item xxx
%\item xxx
%\item xxx
%\end{itemize}
%\end{frame}
%%%%%%%%%%%%%%%%%



\begin{frame}[fragile]{An IDE for Chisel}
\begin{itemize}
\item Eclipse
\item Scala plugin
\item Chisel source included in the project build path
\item But you are not compiling with Eclipse\\ and against the Chisel source
\item Show it
\end{itemize}
\end{frame}

\begin{frame}[fragile]{Chisel in T-CREST}
\begin{itemize}
\item Patmos processor rewritten in Chisel
\begin{itemize}
\item As part of learning Chisel
\item 6.4.2013: Chisel: 996 LoC vs VHDL: 3020 LoC
\item But VHDL was very verbose, with records maybe 2000 LoC
\end{itemize}
\item Memory controller, memory arbiters, IO devices in Chisel
\item One Phd, two master, and one bachelor project:
\begin{itemize}
\item Patmos stack cache
\item Method cache for Patmos -- Chisel was relative easy
\item TDM based memory arbiter -- trouble with Chisel
\item RISC stack cache -- no issues with Chisel
\end{itemize}
\end{itemize}
\end{frame}

\begin{frame}[fragile]{Chisel in Teaching}
\begin{itemize}
\item Using/offering it in Advanced Computer Architecture
\item Spring 2016 all projects have been in Chisel
\item Students pick it up reasonable fast
\item For software engineering students easier than VHDL
\item Issue of \emph{writing a program} instead of describing hardware remains
\end{itemize}
\end{frame}

\begin{frame}[fragile]{Further Information}
\begin{itemize}
\item \url{https://chisel.eecs.berkeley.edu/}
\item Chisel 2.2 Tutorial
\item Getting Started with Chisel
\item \url{http://groups.google.com/group/chisel-users}
\end{itemize}
\end{frame}


\begin{frame}[fragile]{Hello World in Chisel and Examples}
\begin{chisel}
git clone https://github.com/schoeberl/chisel-examples.git
\end{chisel}
\begin{itemize}
\item or download from \url{https://github.com/schoeberl/chisel-examples}
\item This contains a minimal Chisel project with the blinking LED
\item Has ready to use project files for:
\begin{itemize}
\item Altera DE0
\item Altera DE1
\item Altera DE2-115
\item BeMicro
\end{itemize}
\item Plus a simple ALU for HW test and showing Chisel Tester
\item Plus some more examples to explore
\end{itemize}
\end{frame}

\begin{frame}[fragile]{Lab Time}
\begin{itemize}
\item Get a blinking LED working on the FPGA board
\item Clone or download the repository now
\end{itemize}
\begin{chisel}
git clone https://github.com/schoeberl/chisel-examples.git
cd chisel-examples/hello-world
make
\end{chisel}
\begin{itemize}
\item Start Quartus
\item Open the project under chisel-examples/hello-world/quartus/altde2-115
\item Synthesize with the Play button
\item Configure the FPGA
\item Enjoy!
\end{itemize}
\end{frame}

\begin{frame}[fragile]{What is a Minimal Chisel Project?}
\begin{itemize}
\item Scala class (e.g., \code{Hello.scala})
\item Build info in \code{build.sbt} for \code{sbt}:
\end{itemize}
\begin{chisel}
scalaVersion := "2.11.7"

libraryDependencies += "edu.berkeley.cs" %% "chisel" % "latest.release"
\end{chisel}
\begin{itemize}
\item Run the process manually (look into the Makefile)
\end{itemize}
\end{frame}

\begin{frame}[fragile]{Additional Files in the Hello World Example}
\begin{itemize}
\item A \code{Makefile}
\item A Verilog top level file \code{verilog/hello\_top.v}
\begin{itemize}
\item For reset and possible PLLs
\item Can also be in VHDL
\end{itemize}
\item Quartus project files \code{quartus/altde2-115/hello.q\{p|s\}f}
\begin{itemize}
\item List of source files, device and pin assignments
\item plain text files, look into hello.qsf
\end{itemize}
\end{itemize}
\end{frame}

\begin{frame}[fragile]{Change the Design}
\begin{itemize}
\item Setup Eclipse with a Scala project and the Chisel source
\begin{itemize}
\item Optional, use the editor you like most
\end{itemize}
\item Change blinking frequency
\item Rerun the example
\end{itemize}
\end{frame}

\end{document}