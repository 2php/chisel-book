\documentclass[xcolor=pdflatex,dvipsnames,table]{beamer}
\usepackage{epsfig,graphicx}
\usepackage{palatino}
\usepackage{fancybox}
\usepackage{relsize}
\usepackage[procnames]{listings}
\usepackage{hyperref}


% fatter TT font
\renewcommand*\ttdefault{txtt}
% another TT, suggested by Alex
% \usepackage{inconsolata}
% \usepackage[T1]{fontenc} % needed as well?

\usepackage[procnames]{listings}

% shared in slides and book

\lstdefinelanguage{chisel}{
  morekeywords={abstract,case,catch,class,def,%
    do,else,extends,false,final,finally,%
    for,if,implicit,import,match,mixin,%
    new,null,object,override,package,%
    private,protected,requires,return,sealed,%
    super,this,throw,trait,true,try,%
    type,val,var,while,with,yield},
  otherkeywords={=>,<-,<\%,<:,>:,\#,@},
  sensitive=true,
  morecomment=[l]{//},
  morecomment=[n]{/*}{*/},
  morestring=[b]",
  morestring=[b]',
  morestring=[b]"""
}

\usepackage{color}
\definecolor{dkgreen}{rgb}{0,0.6,0}
\definecolor{gray}{rgb}{0.5,0.5,0.5}
\definecolor{mauve}{rgb}{0.58,0,0.82}

% Default settings for code listings
\ifbook
\lstset{%frame=lines,
  language=chisel,
  aboveskip=3mm,
  belowskip=3mm,
  showstringspaces=false,
  columns=fixed, % basewidth=\mybasewidth,
  basicstyle={\small\ttfamily},
  numbers=none,
  numberstyle=\footnotesize,
  % identifierstyle=\color{red},
  breaklines=true,
  breakatwhitespace=true,
  procnamekeys={def, val, var, class, trait, object, extends},
  % procnamestyle=\ttfamily,
  tabsize=2,
  float
}
\else
\lstset{%frame=lines,
  language=chisel,
  aboveskip=3mm,
  belowskip=3mm,
  showstringspaces=false,
  columns=fixed, % basewidth=\mybasewidth,
  basicstyle={\small\ttfamily},
  numbers=none,
  numberstyle=\footnotesize\color{gray},
  % identifierstyle=\color{red},
  keywordstyle=\color{blue},
  commentstyle=\color{dkgreen},
  stringstyle=\color{mauve},
  breaklines=true,
  breakatwhitespace=true,
  procnamekeys={def, val, var, class, trait, object, extends},
  procnamestyle=\ttfamily\color{red},
  tabsize=2,
  float
}
\fi

\lstnewenvironment{chisel}[1][]
{\lstset{language=chisel,#1}}
{}

\newcommand{\shortlist}[1]{{\lstinputlisting[nolol]{#1}}}

\newcommand{\longlist}[3]{{\lstinputlisting[float, caption={#2}, label={#3}, frame=tb, captionpos=b]{#1}}}

\newcommand{\verylonglist}[3]{{\lstinputlisting[caption={#2}, label={#3}, frame=tb, captionpos=b]{#1}}}


\hypersetup{
  linkcolor  = black,
%  citecolor  = blue,
  urlcolor   = blue,
  colorlinks = true,
}

\newcommand{\code}[1]{{\texttt{#1}}}

\beamertemplatenavigationsymbolsempty
\setbeamertemplate{footline}[frame number]

\newcommand{\todo}[1]{{\emph{TODO: #1}}}
\newcommand{\martin}[1]{{\color{blue} Martin: #1}}
\newcommand{\abcdef}[1]{{\color{red} Author2: #1}}

% uncomment following for final submission
%\renewcommand{\todo}[1]{}
%\renewcommand{\martin}[1]{}
%\renewcommand{\author2}[1]{}


\title{Chisel Basic Operations}
\author{Martin Schoeberl}
\date{\today}
\institute{Technical University of Denmark}

\begin{document}

\begin{frame}
\titlepage
\end{frame}

\begin{frame}[fragile]{Chisel Data Types}
\begin{itemize}
\item Data types for values on wires or state elements
\item Raw collection of bits is type \code{Bits}
\item Simple types to represent integer numbers
\begin{itemize}
\item Unsigned and signed
\item Subtype of \code{Bits}
\end{itemize}
\item Little strange way to specify constants
\item Automatic bit width inference
\item Boolean values are of type \code{Bool}, a single bit value
\end{itemize}
\begin{chisel}
UInt(1)
UInt("habcd")
UInt("b0101")
SInt(-5)
Bool(true)
\end{chisel}
\end{frame}

\begin{frame}[fragile]{Chisel Data Types}
\begin{itemize}
\item Bit width can be explicitly specified
\begin{itemize}
\item \code{SInt} will be sign extended
\item \code{UInt} will be zero extended
\end{itemize}
\end{itemize}
\begin{chisel}
UInt(0, 32)
UInt("habcd", 24)
SInt(-5, 16)
Bool(true)
\end{chisel}
\begin{itemize}
\item \emph{Bundles} for a named collection of values
\item \emph{Vecs} for indexable collection of values
\item Chisel data types are different from Scala builtin types (e.g., Scala's \code{Int})
\end{itemize}
\end{frame}

\begin{frame}[fragile]{Bitwise Logical Operations}
\begin{itemize}
\item Bitwise NOT, AND, OR, and XOR
\item Automatic size extension to larger operand
\end{itemize}
\begin{chisel}
val notVal = ~x
val maskOut = x & UInt("b00001111")
val orVal = x | y
val xorVal = x ^ y
\end{chisel}
\begin{itemize}
\item Bit reduction
\item Results in a \code{Bool}
\end{itemize}
\begin{chisel}
andR(x)
orR(x)
xorR(x)
\end{chisel}
\end{frame}

\begin{frame}[fragile]{Arithmetic Operations}
\begin{itemize}
\item Addition, subtraction, multiplication, division, modulos
\item Automatic size extension to larger operand
\end{itemize}
\begin{chisel}
+, -, *, /, %
\end{chisel}
\begin{itemize}
\item Left and right shifts
\item Left shift extend bit width
\item Right shift reduces bit width
\end{itemize}
\begin{chisel}
<<, >>
\end{chisel}
\end{frame}

\begin{frame}[fragile]{Bitfield Manipulations}
\begin{itemize}
\item Extract a single bit
\end{itemize}
\begin{chisel}
val sign = x(31)
\end{chisel}
\begin{itemize}
\item Extract a sub field from end to start position
\end{itemize}
\begin{chisel}
val lowByte = word(7, 0)
\end{chisel}
\begin{itemize}
\item Concatenate bit fields
\end{itemize}
\begin{chisel}
val word = Cat(highByte, lowByte)
\end{chisel}
\end{frame}

\begin{frame}[fragile]{Comparison}
\begin{itemize}
\item The usual operations
\begin{itemize}
\item Unusual equal and unequal operator symbols
\item To keep the original Sala operators usable
\end{itemize}
\item Operands are \code{UInt} and \code{SInt}
\item Can be \code{Bool} for equal and unequal
\item Result is \code{Bool}
\end{itemize}
\begin{chisel}
===, =/=
>, >=, <, <=
\end{chisel}
\end{frame}

\begin{frame}[fragile]{Boolean Logical Operations}
\begin{itemize}
\item Operands and result are \code{Bool}
\item Logical NOT, AND, and OR
\end{itemize}
\begin{chisel}
val notX = !x
val bothTrue = a && b
val orVal = x || y
\end{chisel}
\end{frame}

\begin{frame}[fragile]{Combinational Circuits}
\begin{itemize}
\item Circuit is a graph of nodes
\item A node is a hardware operator with zero or more inputs
\item Textual expression to wire up nodes
\item Named wires with some (unspecified) width
\end{itemize}
\begin{chisel}
(a | b) & ~c
\end{chisel}
\end{frame}

\begin{frame}[fragile]{Combinational Circuits}
\begin{itemize}
\item Simple expressions represent a circuit tree
\item Arbitrary directed acyclic graphs need named subexpressions
\item Using Scala's \code{val} keyword for variables that don't change
\item Referenced multiple times
\end{itemize}
\begin{chisel}
val cond = a & b
val result = (cond & selA) | (!cond & selB)
\end{chisel}
\end{frame}

\begin{frame}[fragile]{Register}
\begin{itemize}
\item State elements
\item Has it's own Chisel type \code{Reg}
\item Positive edge triggered D flip-flop
\item Synchronous reset
\item Clock and reset are \emph{hidden wires}
\end{itemize}
\begin{chisel}
val q = Reg(next = d)
\end{chisel}
\begin{itemize}
\item \code{d} is the input, \code{q} the output
\item Register type is inferred by the input (d) type
\end{itemize}
\end{frame}

\begin{frame}[fragile]{Register}
\begin{itemize}
\item Reset value as \code{init} parameter on definition
\end{itemize}
\begin{chisel}
val initReg = Reg(init = UInt(0, 8))
\end{chisel}
\begin{itemize}
\item With this forward declaration we later assign the next value
\end{itemize}
\begin{chisel}
initReg := initReg + UInt(1)
\end{chisel}
\begin{itemize}
\item A register can also be defined within an expression
\end{itemize}
\begin{chisel}
val risingEdge = d & !Reg(next = d)
\end{chisel}
\end{frame}

\begin{frame}[fragile]{Multiplexer}
\begin{itemize}
\item So common: a component provided by Chisel
\item Could be implemented with conditional updates
\item Automagical type selection on input types
\end{itemize}
\begin{chisel}
val selection = Mux(cond, trueVal, falseVal)
\end{chisel}
\end{frame}

\begin{frame}[fragile]{A Small Circuit}
\begin{itemize}
\item Our Chisel knowledge is complete enough\\ to implement any digital circuit
\item Maybe not in the most elegant way ;-)
\item A counter is a simple basic component
\end{itemize}
\begin{chisel}
  val cntReg = Reg(init = UInt(0, 8))

  cntReg := Mux(cntReg === UInt(100),
    UInt(0), cntReg + UInt(1))
\end{chisel}
\end{frame}

\begin{frame}[fragile]{The Complete Counter Module}
\begin{chisel}
class Count extends Module {
  val io = new Bundle {
    val cnt = UInt(OUTPUT, 8)
  }

  val cntReg = Reg(init = UInt(0, 8))

  cntReg := Mux(cntReg === UInt(100),
    UInt(0), cntReg + UInt(1))

  io.cnt := cntReg
}
\end{chisel}
\end{frame}

\begin{frame}[fragile]{Data Aggregation}
\begin{itemize}
\item A \code{Bundle} groups several named fields
\item Like a C struct
\item \code{Vec} is a vector of elements with the same type
\item Can be arbitrary mixed
\end{itemize}
\begin{chisel}
class AluFields extends Bundle {
  val function = UInt(2)
  val inputA = UInt(8)
  val inputB = UInt(8)
  val result = UInt(8)
}
\end{chisel}
\end{frame}

\begin{frame}[fragile]{Vectors}
\begin{itemize}
\item Indexable vector of elements
\item Elements can be Chisel basic elements, or bundles
\item Type is specified within the curly brackets
\end{itemize}
\begin{chisel}
val myVec = Vec.fill(3) { SInt(width = 10) }
val y = myVec(2)
myVec(0) := SInt(-3)
\end{chisel}
\end{frame}

\begin{frame}[fragile]{Ports}
\begin{itemize}
\item Ports used to connect modules
\item Ports are bundles with directions
\end{itemize}
\begin{chisel}
class AluIO extends Bundle {
  val function = UInt(INPUT, 2)
  val inputA = UInt(INPUT, 8)
  val inputB = UInt(INPUT, 8)
  val result = UInt(OUTPUT, 8)
}
\end{chisel}
\end{frame}

\begin{frame}[fragile]{Port Directions}
\begin{itemize}
\item Can be assigned at instantiation
\end{itemize}
\begin{chisel}
class ExecuteIO extends Bundle {
  val dec = new DecodeExecute().asInput
  val mem = new ExecuteMemory().asOutput
}
\end{chisel}
\begin{chisel}
\end{chisel}
\end{frame}

\begin{frame}[fragile]{Port Directions}
\begin{itemize}
\item Can be reversed with the \code{flip} operation
\item Convenient to have one bundle definition working as source
and destination used between two modules
\end{itemize}
\begin{chisel}
class Channel extends Bundle {
  val data = UInt(INPUT, 32)
  val ready = Bool(OUTPUT)
  val valid = Bool(INPUT)
}

class ChannelUsage extends Bundle {
  val input = new Channel()
  val output = new Channel().flip()
}
\end{chisel}
\end{frame}

\begin{frame}[fragile]{Modules}
\begin{itemize}
\item Modules are used to organize the
\item Similar to VHDL components (entity/architecture)
\item Inherits from \code{Module}
\item Circuit description in the constructor
\item Interface is a Bundle stored in the field \code{io}
\end{itemize}
\begin{chisel}
class Adder extends Module {
  val io = new Bundle {
    val a = UInt(INPUT, 4)
    val b = UInt(INPUT, 4)
    val result = UInt(OUTPUT, 4)
  }

  val addVal = io.a + io.b
  io.result := addVal
}
\end{chisel}
\end{frame}

\begin{frame}[fragile]{Module Usage}
\begin{itemize}
\item Creation with \code{new} and wrapped into a \code{Module}
\item Interface port via the \code{io} field
\item Note the assignment operator \code{:=} on \code{io} fields
\end{itemize}
\begin{chisel}
val adder = Module(new Adder())
adder.io.a := ina
adder.io.b := inb
val result = adder.io.result
\end{chisel}
\end{frame}

\begin{frame}[fragile]{Next Topics -- Some Order}
\begin{itemize}
\item See notes on the back of the slide printout
\item Chisel Background and some Scala (constructor code...)
\item Testing 1
\item when/otherwise, case statements
\item Some more code examples, e.g. counter then with when statement
\item Functions
\item Function examples for combinational circuits (e.g. ALU)
\item Functions with state (rising edge, counter,...)
\item FSM
\item Wave form debugging
\item printf debugging
\item Memories
\item Modules and structure
\item Bulk connections
\item Scala for circuit creation (``scripting'' hardware generation)
\end{itemize}
\end{frame}

\begin{frame}[fragile]{Functions}
\begin{itemize}
\item Section 6, but that might come later
\item xxx
\item xxx
\end{itemize}
\begin{chisel}
code
\end{chisel}
\end{frame}

\begin{frame}[fragile]{Title}
\begin{itemize}
\item xxx
\item xxx
\item xxx
\end{itemize}
\begin{chisel}
code
\end{chisel}
\end{frame}


\begin{frame}[fragile]{More Chisel Example Code}
\begin{itemize}
\item The time-predictable processor Patmos
\item An SRAM controller for the DE2-115 board
\item An SSRAM controller
\item An UART
\item A memory arbiter
\item Caches
\item ...
\item \url{https://github.com/t-crest/patmos}
\end{itemize}
\end{frame}


\begin{frame}[fragile]{More Chisel Documentation}
\begin{itemize}
\item Started a textbook on ``Digital Design with Chisel''
\item Considered a work-in-progress (V 0.01 ;-)
\item \url{https://github.com/schoeberl/chisel-book}
\item Feedback is welcome
\item Contains all the slides
\end{itemize}
\end{frame}

\begin{frame}[fragile]{Chisel Tutorial from UCB}
\begin{itemize}
\item Collection of small exercises
\item Only in simulation, no hardware required (+/-)
\item All examples in \emph{one} design
\begin{itemize}
\item Results in a little bit more complex setup
\end{itemize}
\item Needs a Internet connection
\begin{itemize}
\item  Tests against latest Chisel version
\end{itemize}
\end{itemize}
\end{frame}

\begin{frame}[fragile]{Chisel Tutorial}
\begin{itemize}
\item Get the tutorial
\end{itemize}
\begin{chisel}
git clone https://github.com/ucb-bar/chisel-tutorial.git
cd chisel-tutorial
\end{chisel}
\begin{itemize}
\item Test the installation with a Hello World
\end{itemize}
\begin{chisel}
cd hello
make
\end{chisel}
\begin{itemize}
\item May take some time
\end{itemize}
\end{frame}

\begin{frame}[fragile]{Very Minimal Hello World}
\begin{chisel}
class Hello extends Module {
  val io = new Bundle { 
    val out = UInt(OUTPUT, 8)
  }
  io.out := UInt(42)
}
\end{chisel}
\begin{itemize}
\item Produces hardware for a single constant
\end{itemize}
\end{frame}

\begin{frame}[fragile]{Testing the Minimal Hello World}
\begin{chisel}
class HelloTests(c: Hello) extends Tester(c) {
  step(1)
  expect(c.io.out, 42)
}
\end{chisel}
\begin{itemize}
\item Drive the simulation with \code{step(1)}, which is a single clock tick
\item Test output against expected value
\end{itemize}
\end{frame}

\begin{frame}[fragile]{Tutorial Problems}
\begin{chisel}
cd problems
make Mux2.out
\end{chisel}
\begin{itemize}
\item This example should already work
\item Read the hardware description and test code
\end{itemize}
\end{frame}

\begin{frame}[fragile]{Tutorial Problems}
\begin{chisel}
make Mux4.out
\end{chisel}
\begin{itemize}
\item Keep sbt running
\item Fix the Mux4 component so that the tests complete
\end{itemize}
\end{frame}

%%%%%%%%%%%%%%%%
\begin{frame}[fragile]{Title}
\begin{chisel}
code
\end{chisel}
\begin{itemize}
\item xxx
\item xxx
\item xxx
\end{itemize}
\end{frame}

\begin{frame}[fragile]{Title}
\begin{itemize}
\item xxx
\item xxx
\item xxx
\item xxx
\item xxx
\end{itemize}
\end{frame}
%%%%%%%%%%%%%%%%



\end{document}