\documentclass[xcolor=pdflatex,dvipsnames,table]{beamer}
\usepackage{epsfig,graphicx}
\usepackage{palatino}
\usepackage{fancybox}
\usepackage{relsize}
\usepackage[procnames]{listings}
\usepackage{hyperref}


% fatter TT font
\renewcommand*\ttdefault{txtt}
% another TT, suggested by Alex
% \usepackage{inconsolata}
% \usepackage[T1]{fontenc} % needed as well?

\usepackage[procnames]{listings}

% shared in slides and book

\lstdefinelanguage{chisel}{
  morekeywords={abstract,case,catch,class,def,%
    do,else,extends,false,final,finally,%
    for,if,implicit,import,match,mixin,%
    new,null,object,override,package,%
    private,protected,requires,return,sealed,%
    super,this,throw,trait,true,try,%
    type,val,var,while,with,yield},
  otherkeywords={=>,<-,<\%,<:,>:,\#,@},
  sensitive=true,
  morecomment=[l]{//},
  morecomment=[n]{/*}{*/},
  morestring=[b]",
  morestring=[b]',
  morestring=[b]"""
}

\usepackage{color}
\definecolor{dkgreen}{rgb}{0,0.6,0}
\definecolor{gray}{rgb}{0.5,0.5,0.5}
\definecolor{mauve}{rgb}{0.58,0,0.82}

% Default settings for code listings
\ifbook
\lstset{%frame=lines,
  language=chisel,
  aboveskip=3mm,
  belowskip=3mm,
  showstringspaces=false,
  columns=fixed, % basewidth=\mybasewidth,
  basicstyle={\small\ttfamily},
  numbers=none,
  numberstyle=\footnotesize,
  % identifierstyle=\color{red},
  breaklines=true,
  breakatwhitespace=true,
  procnamekeys={def, val, var, class, trait, object, extends},
  % procnamestyle=\ttfamily,
  tabsize=2,
  float
}
\else
\lstset{%frame=lines,
  language=chisel,
  aboveskip=3mm,
  belowskip=3mm,
  showstringspaces=false,
  columns=fixed, % basewidth=\mybasewidth,
  basicstyle={\small\ttfamily},
  numbers=none,
  numberstyle=\footnotesize\color{gray},
  % identifierstyle=\color{red},
  keywordstyle=\color{blue},
  commentstyle=\color{dkgreen},
  stringstyle=\color{mauve},
  breaklines=true,
  breakatwhitespace=true,
  procnamekeys={def, val, var, class, trait, object, extends},
  procnamestyle=\ttfamily\color{red},
  tabsize=2,
  float
}
\fi

\lstnewenvironment{chisel}[1][]
{\lstset{language=chisel,#1}}
{}

\newcommand{\shortlist}[1]{{\lstinputlisting[nolol]{#1}}}

\newcommand{\longlist}[3]{{\lstinputlisting[float, caption={#2}, label={#3}, frame=tb, captionpos=b]{#1}}}

\newcommand{\verylonglist}[3]{{\lstinputlisting[caption={#2}, label={#3}, frame=tb, captionpos=b]{#1}}}


\hypersetup{
  linkcolor  = black,
%  citecolor  = blue,
  urlcolor   = blue,
  colorlinks = true,
}

\newcommand{\code}[1]{{\texttt{#1}}}

\beamertemplatenavigationsymbolsempty
\setbeamertemplate{footline}[frame number]

\newcommand{\todo}[1]{{\emph{TODO: #1}}}
\newcommand{\martin}[1]{{\color{blue} Martin: #1}}
\newcommand{\abcdef}[1]{{\color{red} Author2: #1}}

% uncomment following for final submission
%\renewcommand{\todo}[1]{}
%\renewcommand{\martin}[1]{}
%\renewcommand{\author2}[1]{}


\title{Customized Circuit Generation}
\author{Martin Schoeberl}
\date{\today}
\institute{Technical University of Denmark}

\begin{document}

\begin{frame}
\titlepage
\end{frame}

\begin{frame}[fragile]{Scala List for Enumeration}
\begin{chisel}
  val empty :: full :: Nil = Enum(UInt(), 2)
\end{chisel}
\begin{itemize}
\item Can be used in wires and registers
\item Symbols for a state machine
\end{itemize}
\end{frame}

\begin{frame}[fragile]{Finite State Machine}
\begin{chisel}
  val empty :: full :: Nil = Enum(UInt(), 2)
  val stateReg = Reg(init = empty)
  val dataReg = Reg(init = Bits(0, size))

  when(stateReg === empty) {
    when(io.enq.write) {
      stateReg := full
      dataReg := io.enq.din
    }
  }.elsewhen(stateReg === full) {
    when(io.deq.read) {
      stateReg := empty
    }
  }
\end{chisel}
\begin{itemize}
\item A simple buffer for a bubble FIFO
\end{itemize}
\end{frame}

\begin{frame}[fragile]{Parameterization}
\begin{chisel}
class ParamChannel(n: Int) extends Bundle {
  val data = UInt(INPUT, n)
  val ready = Bool(OUTPUT)
  val valid = Bool(INPUT)
}

val ch32 = new ParamChannel(32)
\end{chisel}
\begin{itemize}
\item Bundles and modules can be parametrized
\item Pass a parameter in the constructor
\end{itemize}

\end{frame}
\begin{frame}[fragile]{A Module with a Parameter}
\begin{chisel}
class ParamAdder(n: Int) extends Module {
  val io = new Bundle {
    val a = UInt(INPUT, n)
    val b = UInt(INPUT, n)
    val result = UInt(OUTPUT, n)
  }

  val addVal = io.a + io.b
  io.result := addVal
}

val add8 = Module(new ParamAdder(8))
\end{chisel}
\begin{itemize}
\item Parameter can also be a Chisel type
\item Can also be a generic type:
\item \code{class Mod[T <: Bits](param: T) extends...}
\end{itemize}
\end{frame}

\begin{frame}[fragile]{Scala \code{for} Loop for Circuit Generation}
\begin{chisel}
val shiftReg = Reg(init = UInt(0, 8))

shiftReg(0) := inVal

for (i <- 1 until 8) {
  shiftReg(i) := shiftReg(i-1)
}
\end{chisel}
\begin{itemize}
\item \code{for} is Scala
\item This loop generates several connections
\item The connections are parallel hardware
\end{itemize}
\end{frame}

\begin{frame}[fragile]{Conditional Circuit Generation}
\begin{chisel}
class Base extends Module { val io = new Bundle() }
class VariantA extends Base { }
class VariantB extends Base { }

val m = if (useA) Module(new VariantA())
        else Module(new VariantB())
\end{chisel}
\begin{itemize}
\item \code{if} and \code{else} is Scala
\item \code{if} is an expression that returns a value
\begin{itemize}
\item Like ``\code{cond ? a : b;}'' in C and Java
\end{itemize}
\item This is not a hardware multiplexer
\item Decides which module to generate
\item Could even read an XML file for the configuration
\end{itemize}
\end{frame}

\begin{frame}[fragile]{Combinational (Truth) Table Generation}
\begin{chisel}
val arr = new Array[Bits](length)
for (i <- 0 until length) {
  arr(i) = ...
}
val rom = Vec[Bits](arr)
\end{chisel}
\begin{itemize}
\item Generate a table in a Scala array
\item Use that array as input for a Chisel \code{Vec}
\item Generates a logic table at hardware construction time
\end{itemize}
\end{frame}

\begin{frame}[fragile]{Ideas for Runtime Table Generation}
\begin{itemize}
\item Assembler in Scala/Java generates the boot ROM
\item Table with a \code{sin} function
\item Binary to BCD conversion
\item Schedule table for a TDM based network-on-chip
\item 
\item More ideas?
\end{itemize}
\end{frame}

\begin{frame}[fragile]{Memory}
\begin{chisel}
val mem = Mem(Bits(width = 8), size)

// write
when(wrEna) {
  mem(wrAddr) := wrData
}

// read
val rdAddrReg = Reg(next = rdAddr)
rdData := mem(rdAddrReg)
\end{chisel}
\begin{itemize}
\item Write is synchronous
\item Read can be asynchronous or synchronous
\item But there are no asynchronous memories in an FPGA
\end{itemize}
\end{frame}


\begin{frame}[fragile]{Lab Session}
\begin{itemize}
\item Run and explore the UART example (e.g., with gtkterm)
\begin{itemize}
\item In folder examples, package \code{uart}
\item \code{make uart}
\end{itemize}
\item Combine the UART with a blinking LED
\begin{itemize}
\item Write a `0' and `1' out on blink off and on
\end{itemize}
\item Write repeated ``0..9'' as fast as possible 
\end{itemize}
\end{frame}

\begin{frame}[fragile]{Links to the Examples}
\begin{itemize}
\item Example code
\end{itemize}
\begin{chisel}
https://github.com/schoeberl/chisel-examples.git
\end{chisel}
\begin{itemize}
\item Slides
\end{itemize}
\begin{chisel}
https://github.com/schoeberl/chisel-book.git
\end{chisel}
\end{frame}

\begin{frame}[fragile]{Feedback}
\begin{itemize}
\item Plans on using Chisel?
\item What went well?
\item What was not so good?
\item How can this Chisel course be improved?
\item
\item Would be happy to receive an email: \code{masca@dtu.dk}
\end{itemize}
\end{frame}

\end{document}

%%%%%%%%%%%%%%%%
\begin{frame}[fragile]{Title}
\begin{chisel}
code
\end{chisel}
\begin{itemize}
\item xxx
\item xxx
\item xxx
\end{itemize}
\end{frame}

\begin{frame}[fragile]{Title}
\begin{itemize}
\item xxx
\item xxx
\item xxx
\item xxx
\item xxx
\end{itemize}
\end{frame}
%%%%%%%%%%%%%%%%

