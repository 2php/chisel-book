\documentclass[xcolor=pdflatex,dvipsnames,table]{beamer}
\usepackage{epsfig,graphicx}
\usepackage{palatino}
\usepackage{fancybox}
\usepackage{relsize}
\usepackage[procnames]{listings}
\usepackage{hyperref}


% fatter TT font
\renewcommand*\ttdefault{txtt}
% another TT, suggested by Alex
% \usepackage{inconsolata}
% \usepackage[T1]{fontenc} % needed as well?

\usepackage[procnames]{listings}

\newif\ifbook
% shared in slides and book

\lstdefinelanguage{chisel}{
  morekeywords={abstract,case,catch,class,def,%
    do,else,extends,false,final,finally,%
    for,if,implicit,import,match,mixin,%
    new,null,object,override,package,%
    private,protected,requires,return,sealed,%
    super,this,throw,trait,true,try,%
    type,val,var,while,with,yield},
  otherkeywords={=>,<-,<\%,<:,>:,\#,@},
  sensitive=true,
  morecomment=[l]{//},
  morecomment=[n]{/*}{*/},
  morestring=[b]",
  morestring=[b]',
  morestring=[b]"""
}

\usepackage{color}
\definecolor{dkgreen}{rgb}{0,0.6,0}
\definecolor{gray}{rgb}{0.5,0.5,0.5}
\definecolor{mauve}{rgb}{0.58,0,0.82}

% Default settings for code listings
\ifbook
\lstset{%frame=lines,
  language=chisel,
  aboveskip=3mm,
  belowskip=3mm,
  showstringspaces=false,
  columns=fixed, % basewidth=\mybasewidth,
  basicstyle={\small\ttfamily},
  numbers=none,
  numberstyle=\footnotesize,
  % identifierstyle=\color{red},
  breaklines=true,
  breakatwhitespace=true,
  procnamekeys={def, val, var, class, trait, object, extends},
  % procnamestyle=\ttfamily,
  tabsize=2,
  float
}
\else
\lstset{%frame=lines,
  language=chisel,
  aboveskip=3mm,
  belowskip=3mm,
  showstringspaces=false,
  columns=fixed, % basewidth=\mybasewidth,
  basicstyle={\small\ttfamily},
  numbers=none,
  numberstyle=\footnotesize\color{gray},
  % identifierstyle=\color{red},
  keywordstyle=\color{blue},
  commentstyle=\color{dkgreen},
  stringstyle=\color{mauve},
  breaklines=true,
  breakatwhitespace=true,
  procnamekeys={def, val, var, class, trait, object, extends},
  procnamestyle=\ttfamily\color{red},
  tabsize=2,
  float
}
\fi

\lstnewenvironment{chisel}[1][]
{\lstset{language=chisel,#1}}
{}

\newcommand{\shortlist}[1]{{\lstinputlisting[nolol]{#1}}}

\newcommand{\longlist}[3]{{\lstinputlisting[float, caption={#2}, label={#3}, frame=tb, captionpos=b]{#1}}}

\newcommand{\verylonglist}[3]{{\lstinputlisting[caption={#2}, label={#3}, frame=tb, captionpos=b]{#1}}}


\hypersetup{
  linkcolor  = black,
%  citecolor  = blue,
  urlcolor   = blue,
  colorlinks = true,
}

\newcommand{\code}[1]{{\texttt{#1}}}

\beamertemplatenavigationsymbolsempty
\setbeamertemplate{footline}[frame number]

\newcommand{\todo}[1]{{\emph{TODO: #1}}}
\newcommand{\martin}[1]{{\color{blue} Martin: #1}}
\newcommand{\abcdef}[1]{{\color{red} Author2: #1}}

% uncomment following for final submission
%\renewcommand{\todo}[1]{}
%\renewcommand{\martin}[1]{}
%\renewcommand{\author2}[1]{}


\title{Digital Design in the 21st Century: Chisel}
\author{Martin Schoeberl}
\date{\today}
\institute{Technical University of Denmark}

\begin{document}

\begin{frame}
\titlepage
\end{frame}

\begin{frame}[fragile]{Motivating Example: Lipsi, a Tiny Processor}
\begin{itemize}
\item Show in Eclipse
\end{itemize}
\end{frame}

\begin{frame}[fragile]{Motivating Example: Lipsi}
\begin{itemize}
\item Small utility processor Lipsi
\item Hardware described in Chisel
\item Tester in Chisel/Scala
\item Assembler in Scala (core case statement about 20 lines)
\item Testing:
\begin{itemize}
\item Some self testing assembler programs
\item Additionally comparing with a software simulator written in Scala
\end{itemize}
\item All in a single (dual) programming language and a single program!
\item All coded and tested in 14 hours
\end{itemize}
\end{frame}

\begin{frame}[fragile]{Goals for this Intro}
\begin{itemize}
\item Get an idea what Chisel is
\begin{itemize}
\item Will show you code snippets
\end{itemize}
\item A first high level view of the main features of Chisel
\item Reconsider how to describe hardware
\item Some experience report from usage at DTU
\item Pointers to more information
\end{itemize}
\end{frame}

%\begin{frame}[fragile]{Talk abstract}
%
%Date: Tu 19/04/2016, 11:00-12:00
%Room: 123/322
%
%Title: Hardware Design in the 21st Century: with the Object Oriented
%and Functional Language Chisel
%
%Chisel is a hardware construction language implemented as a
%domain specific language in Scala. Therefore, the full power of
%a modern programming language is available to describe hardware
%and, more important, hardware generators. Chisel has been developed
%at UC Berkeley and successfully used for several tape outs of RISC-V.
%Here at DTU we used Chisel in the T-CREST project and in teaching
%advanced computer architecture. Besides presenting small code
%examples in Chisel I will report on experiences on using Chisel in
%the t-CREST project and in teaching.
%
%Martin Schoeberl
%\end{frame}

\begin{frame}[fragile]{Chisel}
\begin{itemize}
\item A hardware \emph{construction} language
\begin{itemize}
\item Constructing Hardware In a Scala Embedded Language
\item If it compiles, it is synthesisable hardware 
\item Say goodby to your unintended latches
\end{itemize}
\item Single source two targets
\begin{itemize}
\item C based cycle accurate simulator
\item Verilog for synthesis
\end{itemize}
\item Embedded in Scala
\begin{itemize}
\item Full power of Scala available
\item But to start with, no Scala knowledge needed
\end{itemize}
\item Developed at UC Berkeley
\end{itemize}
\end{frame}

\begin{frame}[fragile]{Some Notes on Scala}
\begin{itemize}
\item Object oriented
\item Functional
\item Strongly typed
\begin{itemize}
\item With very good type inference
\end{itemize}
\item Could be seen as Java++
\item Compiled to the JVM
\item Good Java interoperability
\begin{itemize}
\item Many libraries available
\end{itemize}
\end{itemize}
\end{frame}

\begin{frame}[fragile]{Chisel vs. Scala}
\begin{itemize}
\item A Chisel hardware description is a Scala program
\item Chisel is a Scala library
\item When the program is executed it generates hardware
\item Chisel is a so-called \emph{embedded domain-specific language}
\end{itemize}
\end{frame}

\begin{frame}[fragile]{Expressions are Combinational Circuits}
\begin{chisel}
(a | b) & ~(c ^ d)

val addVal = a + b
val orVal = a | b
val boolVal = a >= b
\end{chisel}
\begin{itemize}
\item The usual operations 
\item Simple name assignment with val
\item Width inference
\item Type inference
\item Types: Bits, UInt, SInt, Bool
\end{itemize}
\end{frame}

\begin{frame}[fragile]{Registers}
\begin{chisel}
val cntReg = Reg(init = UInt(0, 25))

cntReg := cntReg + UInt(1)
\end{chisel}
\begin{itemize}
\item Type inferred by initial value (= reset value)
\item No need to specify a clock or reset signal
\end{itemize}
\begin{itemize}
\item Also definition with an input signal connected:
\end{itemize}
\begin{chisel}
val r = Reg(next = nextVal) 
\end{chisel}
\end{frame}


\begin{frame}[fragile]{IO Ports}
\begin{chisel}
class Channel extends Bundle {
  val data = Bits(INPUT, 8)
  val ready = Bool(OUTPUT)
  val valid = Bool(INPUT)
}
\end{chisel}
\begin{itemize}
\item Ports are Bundles with directions
\item Direction can also be assigned at instantiation:
\end{itemize}
\begin{chisel}
class ExecuteIO extends Bundle {
  val dec = new DecodeExecute().asInput
  val mem = new ExecuteMemory().asOutput
}
\end{chisel}
\end{frame}

\begin{frame}[fragile]{Modules}
\begin{chisel}
class Adder extends Module {
  val io = new Bundle {
    val a = UInt(INPUT, 4)
    val b = UInt(INPUT, 4)
    val result = UInt(OUTPUT, 4)
  }

  val addVal = io.a + io.b
  io.result := addVal
}
\end{chisel}
\begin{itemize}
\item Organization of components
\item IO ports defined as a Bundle named \code{io}
\item Created (instantiated) with:
\end{itemize}
\begin{chisel}
val adder = Module(new Adder())
\end{chisel}
\end{frame}

\begin{frame}[fragile]{Hello World in Chisel}
\begin{chisel}
class Hello extends Module {
  val io = new Bundle {
    val led = UInt(OUTPUT, 1)
  }
  val CNT_MAX = UInt(20000000 / 2 - 1);
  
  val cntReg = Reg(init = UInt(0, 25))
  val blkReg = Reg(init = UInt(0, 1))

  cntReg := cntReg + UInt(1)
  when(cntReg === CNT_MAX) {
    cntReg := UInt(0)
    blkReg := ~blkReg
  }
  io.led := blkReg
}
\end{chisel}
\end{frame}

\begin{frame}[fragile]{Connections}
\begin{itemize}
\item Simple connections just with assignments, e.g.,
\begin{chisel}
  adder.io.a := ina
  adder.io.b := inb
\end{chisel}
\item Automatic bulk connections between components
\begin{chisel}
  dec.io <> exe.io
  mem.io <> exe.io
\end{chisel}
\end{itemize}
\end{frame}

\begin{frame}[fragile]{Generic Components}
\begin{chisel}
val c = Mux(cond, a, b)
\end{chisel}
\begin{itemize}
\item This is a multiplexer
\item Inputs a and b can be any type
\end{itemize}
\end{frame}



\begin{frame}[fragile]{Chisel Data Types}
\begin{itemize}
\item Data types for values on wires or state elements
\item Raw collection of bits is type \code{Bits}
\item Simple types to represent integer numbers
\begin{itemize}
\item Unsigned and signed
\item Subtype of \code{Bits}
\end{itemize}
\item Little strange way to specify constants
\item Automatic bit width inference
\item Boolean values are of type \code{Bool}, a single bit value
\end{itemize}
\begin{chisel}
UInt(1)
UInt("habcd")
UInt("b0101")
SInt(-5)
Bool(true)
\end{chisel}
\end{frame}

\begin{frame}[fragile]{Chisel Data Types}
\begin{itemize}
\item Bit width can be explicitly specified
\begin{itemize}
\item \code{SInt} will be sign extended
\item \code{UInt} will be zero extended
\end{itemize}
\end{itemize}
\begin{chisel}
UInt(0, 32)
UInt("habcd", 24)
SInt(-5, 16)
Bool(true)
\end{chisel}
\begin{itemize}
\item \emph{Bundles} for a named collection of values
\item \emph{Vecs} for indexable collection of values
\item Chisel data types are different from Scala builtin types (e.g., Scala's \code{Int})
\end{itemize}
\end{frame}

\begin{frame}[fragile]{Bitwise Logical Operations}
\begin{itemize}
\item Bitwise NOT, AND, OR, and XOR
\item Automatic size extension to larger operand
\end{itemize}
\begin{chisel}
val notVal = ~x
val maskOut = x & UInt("b00001111")
val orVal = x | y
val xorVal = x ^ y
\end{chisel}
\begin{itemize}
\item Bit reduction
\item Results in a \code{Bool}
\end{itemize}
\begin{chisel}
andR(x)
orR(x)
xorR(x)
\end{chisel}
\end{frame}

\begin{frame}[fragile]{Arithmetic Operations}
\begin{itemize}
\item Addition, subtraction, multiplication, division, modulos
\item Automatic size extension to larger operand
\end{itemize}
\begin{chisel}
+, -, *, /, %
\end{chisel}
\begin{itemize}
\item Left and right shifts
\item Left shift extends bit width
\item Right shift reduces bit width
\end{itemize}
\begin{chisel}
<<, >>
\end{chisel}
\end{frame}

\begin{frame}[fragile]{Bitfield Manipulations}
\begin{itemize}
\item Extract a single bit
\end{itemize}
\begin{chisel}
val sign = x(31)
\end{chisel}
\begin{itemize}
\item Extract a sub field from end to start position
\end{itemize}
\begin{chisel}
val lowByte = word(7, 0)
\end{chisel}
\begin{itemize}
\item Concatenate bit fields
\end{itemize}
\begin{chisel}
val word = Cat(highByte, lowByte)
\end{chisel}
\end{frame}

\begin{frame}[fragile]{Comparison}
\begin{itemize}
\item The usual operations
\begin{itemize}
\item Unusual equal and unequal operator symbols
\item To keep the original Sala operators usable
\end{itemize}
\item Operands are \code{UInt} and \code{SInt}
\item Operands can be \code{Bool} for equal and unequal
\item Result is \code{Bool}
\end{itemize}
\begin{chisel}
===, =/=
>, >=, <, <=
\end{chisel}
\end{frame}

\begin{frame}[fragile]{Boolean Logical Operations}
\begin{itemize}
\item Operands and result are \code{Bool}
\item Logical NOT, AND, and OR
\end{itemize}
\begin{chisel}
val notX = !x
val bothTrue = a && b
val orVal = x || y
\end{chisel}
\end{frame}

\begin{frame}[fragile]{Combinational Circuits}
\begin{itemize}
\item Circuit is a graph of nodes
\item A node is a hardware operator with zero or more inputs
\item Textual expression to wire up nodes
\item Named wires with some (unspecified) width
\end{itemize}
\begin{chisel}
(a | b) & ~c
\end{chisel}
\end{frame}

\begin{frame}[fragile]{Combinational Circuits}
\begin{itemize}
\item Simple expressions represent a circuit tree
\item Arbitrary directed acyclic graphs need named subexpressions
\item Using Scala's \code{val} keyword for variables that don't change
\item Referenced multiple times
\end{itemize}
\begin{chisel}
val cond = a & b
val result = (cond & selA) | (!cond & selB)
\end{chisel}
\end{frame}

\begin{frame}[fragile]{Register}
\begin{itemize}
\item State elements
\item Has it's own Chisel type \code{Reg}
\item Positive edge triggered D flip-flop
\item Synchronous reset
\item Clock and reset are \emph{hidden wires}
\end{itemize}
\begin{chisel}
val q = Reg(next = d)
\end{chisel}
\begin{itemize}
\item \code{d} is the input, \code{q} the output
\item Register type is inferred by the input (d) type
\end{itemize}
\end{frame}

\begin{frame}[fragile]{Register}
\begin{itemize}
\item Reset value as \code{init} parameter on definition
\end{itemize}
\begin{chisel}
val initReg = Reg(init = UInt(0, 8))
\end{chisel}
\begin{itemize}
\item With this forward declaration we later assign the next value
\end{itemize}
\begin{chisel}
initReg := initReg + UInt(1)
\end{chisel}
\begin{itemize}
\item A register can also be defined within an expression
\end{itemize}
\begin{chisel}
val risingEdge = d & !Reg(next = d)
\end{chisel}
\end{frame}

\begin{frame}[fragile]{Multiplexer}
\begin{itemize}
\item So common: a component provided by Chisel
\item Could be implemented with conditional updates
\item Automagical type selection on input types
\end{itemize}
\begin{chisel}
val selection = Mux(cond, trueVal, falseVal)
\end{chisel}
\end{frame}

\begin{frame}[fragile]{A Small Circuit}
\begin{itemize}
\item Our Chisel knowledge is complete enough\\ to implement any digital circuit
\item Maybe not in the most elegant way ;-)
\item A counter is a simple basic component
\item The following counts form 0 to 100
\end{itemize}
\begin{chisel}
  val cntReg = Reg(init = UInt(0, 8))

  cntReg := Mux(cntReg === UInt(100),
    UInt(0), cntReg + UInt(1))
\end{chisel}
\end{frame}

\begin{frame}[fragile]{The Complete Counter Module}
\begin{chisel}
class Count extends Module {
  val io = new Bundle {
    val cnt = UInt(OUTPUT, 8)
  }

  val cntReg = Reg(init = UInt(0, 8))

  cntReg := Mux(cntReg === UInt(100),
    UInt(0), cntReg + UInt(1))

  io.cnt := cntReg
}
\end{chisel}
\end{frame}

\begin{frame}[fragile]{Data Aggregation}
\begin{itemize}
\item A \code{Bundle} groups several named fields
\item Like a C struct or VHDL record
\item \code{Vec} is a vector of elements with the same type
\item Can be arbitrary mixed
\end{itemize}
\begin{chisel}
class AluFields extends Bundle {
  val function = UInt(2)
  val inputA = UInt(8)
  val inputB = UInt(8)
  val result = UInt(8)
}
\end{chisel}
\end{frame}

\begin{frame}[fragile]{Vectors}
\begin{itemize}
\item Indexable vector of elements
\item Elements can be Chisel basic elements, or bundles
\item Type is specified as second parameter
\end{itemize}
\begin{chisel}
val myVec = Vec(3, SInt(width = 10))
val y = myVec(2)
myVec(0) := SInt(-3)
\end{chisel}
\begin{itemize}
\item A register file as \code{Reg} of a vector
\end{itemize}
\begin{chisel}
val vecReg = Reg(Vec(32, SInt(width = 32)))
\end{chisel}
\end{frame}

\begin{frame}[fragile]{Ports}
\begin{itemize}
\item Ports used to connect modules
\item Ports are bundles with directions
\end{itemize}
\begin{chisel}
class AluIO extends Bundle {
  val function = UInt(INPUT, 2)
  val inputA = UInt(INPUT, 8)
  val inputB = UInt(INPUT, 8)
  val result = UInt(OUTPUT, 8)
}
\end{chisel}
\end{frame}

\begin{frame}[fragile]{Port Directions}
\begin{itemize}
\item Can be assigned at instantiation
\end{itemize}
\begin{chisel}
class ExecuteIO extends Bundle {
  val dec = new DecodeExecute().asInput
  val mem = new ExecuteMemory().asOutput
}
\end{chisel}
\begin{chisel}
\end{chisel}
\end{frame}

\begin{frame}[fragile]{Port Directions}
\begin{itemize}
\item Can be reversed with the \code{flip} operation
\item Convenient to have one bundle definition working as source
and destination used between two modules
\end{itemize}
\begin{chisel}
class Channel extends Bundle {
  val data = UInt(INPUT, 32)
  val ready = Bool(OUTPUT)
  val valid = Bool(INPUT)
}

class ChannelUsage extends Bundle {
  val input = new Channel()
  val output = new Channel().flip()
}
\end{chisel}
\end{frame}

\begin{frame}[fragile]{Modules}
\begin{itemize}
\item Modules are used to organize the circuit
\item Similar to VHDL components (entity/architecture)
\item A class that inherits from \code{Module}
\item Circuit description in the constructor
\item Interface (port) is a Bundle stored in the field \code{io}
\end{itemize}
\begin{chisel}
class Adder extends Module {
  val io = new Bundle {
    val a = UInt(INPUT, 4)
    val b = UInt(INPUT, 4)
    val result = UInt(OUTPUT, 4)
  }

  val addVal = io.a + io.b
  io.result := addVal
}
\end{chisel}
\end{frame}

\begin{frame}[fragile]{Module Usage}
\begin{itemize}
\item Create with \code{new} and wrap into a \code{Module()}
\item Interface port via the \code{io} field
\item Note the assignment operator \code{:=} on \code{io} fields
\end{itemize}
\begin{chisel}
val adder = Module(new Adder())
adder.io.a := ina
adder.io.b := inb
val result = adder.io.result
\end{chisel}
\end{frame}

\begin{frame}[fragile]{Conditional Assignments}
\begin{itemize}
\item Conditional update of a value
\item Last assignment counts
\item Is basically a multiplexer
\end{itemize}
\begin{chisel}
val v = UInt(5)
when (condition) {
  v := UInt(0)
}

when (c1) { v := UInt(1) }
when (c2) { v := UInt(2) }
\end{chisel}
\end{frame}

\begin{frame}[fragile]{The Counter With a Conditional Update}
\begin{chisel}
class Count extends Module {
  val io = new Bundle {
    val cnt = UInt(OUTPUT, 8)
  }

  val cntReg = Reg(init = UInt(0, 8))

  cntReg := cntReg + UInt(1)
  when (cntReg === UInt(100)) {
    cntReg := UInt(0)
  }

  io.cnt := cntReg
}
\end{chisel}
\end{frame}

\begin{frame}[fragile]{Chained Conditionals}
\begin{itemize}
\item Chain of conditionals with \code{.elsewhen}
\item With an optional \emph{else} path with \code{.otherwise}
\item Note that Scala has \code{if/else}
\begin{itemize}
\item Does NOT result in hardware
\item Are used to conditionally \emph{generate} hardware
\item We will look at this later
\end{itemize}
\item Note the ``.'' at the operators
\end{itemize}
\begin{chisel}
when (c1) { v := UInt(1) }
.elsewhen (c2) { v := UInt(2) }
.otherwise { v := UInt(3) }
\end{chisel}
\end{frame}

\begin{frame}[fragile]{Switch Statement}
\begin{itemize}
\item Series of comparisons
\item Chisel allows combinational logic be updated conditionally 
\item Chisel disallows incomplete specified logic (= latches)
\item Chisel will report a runtime error
\end{itemize}
\begin{chisel}
switch(fn) {
  is(UInt(0)) { result := a + b }
  is(UInt(1)) { result := a - b }
  is(UInt(2)) { result := a | b }
  is(UInt(3)) { result := a & b }
}
\end{chisel}
\end{frame}

\begin{frame}[fragile]{More Chisel Example Code}
\begin{itemize}
\item The time-predictable processor Patmos
\item An SRAM controller for the DE2-115 board
\item An SSRAM controller
\item An UART
\item A memory arbiter
\item Caches
\item ...
\item \url{https://github.com/t-crest/patmos}
\end{itemize}
\end{frame}

\begin{frame}[fragile]{Chisel in the T-CREST Project}
\begin{itemize}
\item T-CREST is a multicore research project (open source)
\item Patmos processor rewritten in Chisel
\begin{itemize}
\item As part of learning Chisel
\item 6.4.2013: Chisel: 996 LoC vs VHDL: 3020 LoC
\item But VHDL was very verbose, with records maybe 2000 LoC
\end{itemize}
\item Memory controller, memory arbiters, IO devices in Chisel
\item One Phd, two master, and one bachelor project:
\begin{itemize}
\item Patmos stack cache
\item Method cache for Patmos -- Chisel was relative easy
\item TDM based memory arbiter -- trouble with Chisel
\item RISC stack cache -- no issues with Chisel
\end{itemize}
\end{itemize}
\end{frame}

\begin{frame}[fragile]{Chisel in Teaching}
\begin{itemize}
\item Using/offering it in Advanced Computer Architecture
\item Spring 2016 and 2017 all projects have been in Chisel
\item Several Bachelor and Master projects
\item Students pick it up reasonable fast
\item For software engineering students easier than VHDL
\item Issue of \emph{writing a program} instead of describing hardware remains
\end{itemize}
\end{frame}

\begin{frame}[fragile]{More Chisel Documentation}
\begin{itemize}
\item Started a textbook on ``Digital Design with Chisel''
\item Considered a work-in-progress (V 0.01 ;-)
\item \url{https://github.com/schoeberl/chisel-book}
\item Feedback is welcome
\item Contains all the slides (including exercises)
\end{itemize}
\end{frame}


\begin{frame}[fragile]{Further Information}
\begin{itemize}
\item \url{https://chisel.eecs.berkeley.edu/}
\item Chisel 2 documentation at \url{https://github.com/schoeberl/chisel2-doc}
\begin{itemize}
\item Chisel 2.2 Tutorial
\item Getting Started with Chisel
\end{itemize}
\item \url{http://groups.google.com/group/chisel-users}
\end{itemize}
\end{frame}
\end{document}