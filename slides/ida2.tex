\documentclass[xcolor=pdflatex,dvipsnames,table]{beamer}
\usepackage{epsfig,graphicx}
\usepackage{palatino}
\usepackage{fancybox}
\usepackage{relsize}
\usepackage[procnames]{listings}
\usepackage{hyperref}


% fatter TT font
\renewcommand*\ttdefault{txtt}
% another TT, suggested by Alex
% \usepackage{inconsolata}
% \usepackage[T1]{fontenc} % needed as well?

\usepackage[procnames]{listings}
\newif\ifbook
% shared in slides and book

\lstdefinelanguage{chisel}{
  morekeywords={abstract,case,catch,class,def,%
    do,else,extends,false,final,finally,%
    for,if,implicit,import,match,mixin,%
    new,null,object,override,package,%
    private,protected,requires,return,sealed,%
    super,this,throw,trait,true,try,%
    type,val,var,while,with,yield},
  otherkeywords={=>,<-,<\%,<:,>:,\#,@},
  sensitive=true,
  morecomment=[l]{//},
  morecomment=[n]{/*}{*/},
  morestring=[b]",
  morestring=[b]',
  morestring=[b]"""
}

\usepackage{color}
\definecolor{dkgreen}{rgb}{0,0.6,0}
\definecolor{gray}{rgb}{0.5,0.5,0.5}
\definecolor{mauve}{rgb}{0.58,0,0.82}

% Default settings for code listings
\ifbook
\lstset{%frame=lines,
  language=chisel,
  aboveskip=3mm,
  belowskip=3mm,
  showstringspaces=false,
  columns=fixed, % basewidth=\mybasewidth,
  basicstyle={\small\ttfamily},
  numbers=none,
  numberstyle=\footnotesize,
  % identifierstyle=\color{red},
  breaklines=true,
  breakatwhitespace=true,
  procnamekeys={def, val, var, class, trait, object, extends},
  % procnamestyle=\ttfamily,
  tabsize=2,
  float
}
\else
\lstset{%frame=lines,
  language=chisel,
  aboveskip=3mm,
  belowskip=3mm,
  showstringspaces=false,
  columns=fixed, % basewidth=\mybasewidth,
  basicstyle={\small\ttfamily},
  numbers=none,
  numberstyle=\footnotesize\color{gray},
  % identifierstyle=\color{red},
  keywordstyle=\color{blue},
  commentstyle=\color{dkgreen},
  stringstyle=\color{mauve},
  breaklines=true,
  breakatwhitespace=true,
  procnamekeys={def, val, var, class, trait, object, extends},
  procnamestyle=\ttfamily\color{red},
  tabsize=2,
  float
}
\fi

\lstnewenvironment{chisel}[1][]
{\lstset{language=chisel,#1}}
{}

\newcommand{\shortlist}[1]{{\lstinputlisting[nolol]{#1}}}

\newcommand{\longlist}[3]{{\lstinputlisting[float, caption={#2}, label={#3}, frame=tb, captionpos=b]{#1}}}

\newcommand{\verylonglist}[3]{{\lstinputlisting[caption={#2}, label={#3}, frame=tb, captionpos=b]{#1}}}


\hypersetup{
  linkcolor  = black,
%  citecolor  = blue,
  urlcolor   = blue,
  colorlinks = true,
}

\newcommand{\code}[1]{{\texttt{#1}}}

\beamertemplatenavigationsymbolsempty
\setbeamertemplate{footline}[frame number]

\newcommand{\todo}[1]{{\emph{TODO: #1}}}
\newcommand{\martin}[1]{{\color{blue} Martin: #1}}
\newcommand{\abcdef}[1]{{\color{red} Author2: #1}}

% uncomment following for final submission
%\renewcommand{\todo}[1]{}
%\renewcommand{\martin}[1]{}
%\renewcommand{\author2}[1]{}


\title{A Little Bit of Scala and\\ Customized Circuit Generation}
\author{Martin Schoeberl}
\date{\today}
\institute{Technical University of Denmark}

\begin{document}

\begin{frame}
\titlepage
\end{frame}

\begin{frame}[fragile]{Chisel and Scala}
\begin{itemize}
\item Chisel is a library written in Scala
\begin{itemize}
\item Import the library with \code{import Chisel.\_}
\end{itemize}
\item Chisel code is Scala code
\item When it is run is \emph{generates} hardware
\begin{itemize}
\item Verilog for synthesize
\item C++ code for simulation
\end{itemize}
\item Chisel is an embedded domain specific language
\item Two languages in one can be a little bit confusing
\end{itemize}
\end{frame}

\begin{frame}[fragile]{Scala}
\begin{itemize}
\item Is object oriented
\item Is functional
\item Strongly typed with very good type inference
\item Runs on the Java virtual machine
\item Can call Java libraries
\item Consider it as Java++
\begin{itemize}
\item Can almost be written like Java
\item With a more lightweight syntax
\end{itemize}
\end{itemize}
\end{frame}

\begin{frame}[fragile]{Scala Values and Variables}
\begin{chisel}
// A value is a constant
val i = 0
// No new assignment; this will not compile
i = 3

// A variable can change the value
var v = "Hello"
v = "Hello World"

// Type usually inferred, but can be declared
var s: String = "abc"
\end{chisel}
\end{frame}

\begin{frame}[fragile]{Simple Loops}
\begin{chisel}
// Loops from 0 to 9
// Automatically creates loop value i
for (i <- 0 until 10) {
  println(i)
}
\end{chisel}
\end{frame}

\begin{frame}[fragile]{Conditions}
\begin{chisel}
for (i <- 0 until 10) {
  if (i%2 == 0) {
    println(i + " is even")
  } else {
    println(i + " is odd")
  }
}
\end{chisel}
\end{frame}

\begin{frame}[fragile]{Scala Arrays and Lists}
\begin{chisel}
// An integer array with 10 elements
val numbers = new Array[Integer](10)
for (i <- 0 until numbers.length) {
  numbers(i) = i*10
}
println(numbers(9))


// List of integers
val list = List(1, 2, 3)
println(list)
// Different form of list construction
val listenum = 'a' :: 'b' :: 'c' :: Nil
println(listenum)
\end{chisel}
\end{frame}


\begin{frame}[fragile]{Scala Classes}
\begin{chisel}
// A simple class
class Example {
  // A field, initialized in the constructor
  var n = 0
  
  // A setter method
  def set(v: Integer) {
    n = v
  }
  
  // Another method
  def print() {
    println(n)
  }
}
\end{chisel}
\end{frame}

\begin{frame}[fragile]{Scala (Singleton) Object}
\begin{chisel}
object Example {}
\end{chisel}
\begin{itemize}
\item For \emph{static} fields and methods
\begin{itemize}
\item Scala has no static fields or methods like Java
\end{itemize}
\item Needed for \code{main}
\item Useful for helper functions
\end{itemize}
\end{frame}

\begin{frame}[fragile]{Singleton Object for the \code{main}}
\begin{chisel}
// A singleton object
object Example {
  
  // The start of a Scala program
  def main(args: Array[String]): Unit = {
    
    val e = new Example()
    e.print()
    e.set(42)
    e.print()
  }
}
\end{chisel}
\begin{itemize}
\item Compile and run it with sbt (or within Eclipse):
\end{itemize}
\begin{chisel}
sbt "run-main simple.Example"
\end{chisel}
\end{frame}

\begin{frame}[fragile]{Scala Build Tool (sbt)}
\begin{itemize}
\item Downloads Scala compiler if needed
\item Downloads dependent libraries (e.g., Chisel)
\item Compiles Scala programs
\item Executes Scala programs
\item Does a lot of magic, maybe too much
\item Compile and run with:
\end{itemize}
\begin{chisel}
sbt "run-main simple.Example"
\end{chisel}
\begin{itemize}
\item Or even just:
\end{itemize}
\begin{chisel}
sbt run
\end{chisel}
\end{frame}

\begin{frame}[fragile]{Chisel in Scala}
\begin{itemize}
\item Chisel components are Scala classes
\item Chisel code is in the constructor
\item Executed at object creation time
\item Builds the network of hardware objects
\item Testers are written in Scala to drive the tests
\end{itemize}
\end{frame}


\begin{frame}[fragile]{A Chisel Tester}
\begin{itemize}
\item Extends class \code{Tester}
\item Has the DUT as parameter
\item Testing code can use all features of Scala
\end{itemize}
\begin{chisel}
class AdderTester(dut: Adder) extends Tester(dut) {

  // Here comes the Chisel/Scala code
  // for the testing
}
\end{chisel}
\end{frame}

\begin{frame}[fragile]{Testing}
\begin{itemize}
\item Set input values with \code{poke}
\item Advance the simulation with \code{step}
\item Read the output values with \code{peek}
\item Compare the values with \code{expect}
\item Can use full power of the general purpose language Scala
\end{itemize}
\end{frame}

\begin{frame}[fragile]{Testing Example}
\begin{chisel}
// Set input values
poke(dut.io.a, 3)
poke(dut.io.b, 4)
// Execute one iteration
step(1)
// Print the result
val res = peek(dut.io.result)
println(res)

// Or compare against expected value
expect(dut.io.result, 7)
\end{chisel}
\end{frame}

% Does not work reliable
%\begin{frame}[fragile]{Poor Mans Debugging}
%\begin{itemize}
%\item Alternative: use \code{printf} in your Chisel components
%\item Executed at each rising edge of the clock
%\end{itemize}
%\begin{chisel}
%code
%\end{chisel}
%\end{frame}

\begin{frame}[fragile]{A Tiny ALU: IO Connection}
\begin{chisel}
class Alu extends Module {
  val io = new Bundle {
    val fn = UInt(INPUT, 2)
    val a = UInt(INPUT, 4)
    val b = UInt(INPUT, 4)
    val result = UInt(OUTPUT, 4)
  }

  // Use shorter variable names
  val fn = io.fn
  val a = io.a
  val b = io.b
\end{chisel}
\end{frame}

\begin{frame}[fragile]{A Tiny ALU: The Function}
\begin{chisel}
  val result = UInt(width = 4)
  // some default value is needed
  result := UInt(0)

  // The ALU selection
  switch(fn) {
    is(UInt(0)) { result := a + b }
    is(UInt(1)) { result := a - b }
    is(UInt(2)) { result := a | b }
    is(UInt(3)) { result := a & b }
  }

  // Output on the LEDs
  io.result := result
}
\end{chisel}
\end{frame}

\begin{frame}[fragile]{Testing the ALU}
\begin{itemize}
\item Compute the expected result in Scala
\end{itemize}
\begin{chisel}
  // This is exhaustive testing,
  // which usually is impossible
  for (a <- 0 to 15) {
    for (b <- 0 to 15) {
      for (op <- 0 to 3) {
        val result =
          op match {
            case 0 => a + b
            case 1 => a - b
            case 2 => a | b
            case 3 => a & b
          }
        val resMask = result & 0x0f
\end{chisel}

\end{frame}

\begin{frame}[fragile]{Testing the ALU}
\begin{itemize}
\item Compare the Scala computed result with the hardware result
\end{itemize}
\begin{chisel}
        poke(dut.io.fn, op)
        poke(dut.io.a, a)
        poke(dut.io.b, b)
        step(1)
        expect(dut.io.result, resMask)
      }
    }
  }
\end{chisel}
\end{frame}

\begin{frame}[fragile]{Generating Wave Forms}
\begin{itemize}
\item Additional parameter \code{--vcd}
\item IO signals and registers are dumped
\item Option \code{--debug} puts all wires into the dump
\item Generates a .vcd file
\item Viewing with gtkwave or ModelSim
\item See the example with \code{make fifo}
\begin{itemize}
\item Show it
\end{itemize}
\item BubbleFifo contains also longer testing code
\end{itemize}
\end{frame}

\begin{frame}[fragile]{Functional Abstraction}
\begin{itemize}
\item Circuits can be encapsulated in functions
\item Each \emph{function call} generates hardware
\item Simple functions can be a single line
\end{itemize}
\begin{chisel}
  def adder(v1: UInt, v2: UInt) = v1 + v2
  
  val add1 = adder(a, b)
  val add2 = adder(c, d)
\end{chisel}
\end{frame}

\begin{frame}[fragile]{More Function Examples}
\begin{itemize}
\item Functions can also contain registers (contain state)
\end{itemize}
\begin{chisel}
  def addSub(add: Bool, a: UInt, b: UInt) =
    Mux(add, a + b, a - b)

  val res = addSub(cond, a, b)

  def rising(d: Bool) = d && !Reg(next = d)

  val edge = rising(cond)
\end{chisel}
\end{frame}

\begin{frame}[fragile]{The Counter as a Function}
\begin{itemize}
\item Longer functions in curly brackets
\item Last value is the return value
\end{itemize}
\begin{chisel}
def counter(n: UInt) = {
  
  val cntReg = Reg(init = UInt(0, 8))
  
  cntReg := cntReg + UInt(1)
  when(cntReg === n) {
    cntReg := UInt(0)
  }
  cntReg
}

val counter100 = counter(UInt(100))
\end{chisel}
\end{frame}


\begin{frame}[fragile]{Functions}
\begin{itemize}
\item Example from Patmos execute stage
\end{itemize}
\begin{chisel}
def alu(func: Bits, op1: UInt, op2: UInt): Bits = {
  val result = UInt(width = DATA_WIDTH)
  // some more lines...
  switch(func) {
    is(FUNC_ADD) { result := sum }
    is(FUNC_SUB) { result := op1 - op2 }
    is(FUNC_XOR) { result := (op1 ^ op2).toUInt }
    // some more lines
  }
  result
}
\end{chisel}
\end{frame}

\begin{frame}[fragile]{Scala List for Enumeration}
\begin{chisel}
  val empty :: full :: Nil = Enum(UInt(), 2)
\end{chisel}
\begin{itemize}
\item Can be used in wires and registers
\item Symbols for a state machine
\end{itemize}
\end{frame}

\begin{frame}[fragile]{Finite State Machine}
\begin{chisel}
  val empty :: full :: Nil = Enum(UInt(), 2)
  val stateReg = Reg(init = empty)
  val dataReg = Reg(init = Bits(0, size))

  when(stateReg === empty) {
    when(io.enq.write) {
      stateReg := full
      dataReg := io.enq.din
    }
  }.elsewhen(stateReg === full) {
    when(io.deq.read) {
      stateReg := empty
    }
  }
\end{chisel}
\begin{itemize}
\item A simple buffer for a bubble FIFO
\end{itemize}
\end{frame}

\begin{frame}[fragile]{Parameterization}
\begin{chisel}
class ParamChannel(n: Int) extends Bundle {
  val data = UInt(INPUT, n)
  val ready = Bool(OUTPUT)
  val valid = Bool(INPUT)
}

val ch32 = new ParamChannel(32)
\end{chisel}
\begin{itemize}
\item Bundles and modules can be parametrized
\item Pass a parameter in the constructor
\end{itemize}

\end{frame}
\begin{frame}[fragile]{A Module with a Parameter}
\begin{chisel}
class ParamAdder(n: Int) extends Module {
  val io = new Bundle {
    val a = UInt(INPUT, n)
    val b = UInt(INPUT, n)
    val result = UInt(OUTPUT, n)
  }

  val addVal = io.a + io.b
  io.result := addVal
}

val add8 = Module(new ParamAdder(8))
\end{chisel}
\begin{itemize}
\item Parameter can also be a Chisel type
\item Can also be a generic type:
\item \code{class Mod[T <: Bits](param: T) extends...}
\end{itemize}
\end{frame}

\begin{frame}[fragile]{Scala \code{for} Loop for Circuit Generation}
\begin{chisel}
val shiftReg = Reg(init = UInt(0, 8))

shiftReg(0) := inVal

for (i <- 1 until 8) {
  shiftReg(i) := shiftReg(i-1)
}
\end{chisel}
\begin{itemize}
\item \code{for} is Scala
\item This loop generates several connections
\item The connections are parallel hardware
\end{itemize}
\end{frame}

\begin{frame}[fragile]{Conditional Circuit Generation}
\begin{chisel}
class Base extends Module { val io = new Bundle() }
class VariantA extends Base { }
class VariantB extends Base { }

val m = if (useA) Module(new VariantA())
        else Module(new VariantB())
\end{chisel}
\begin{itemize}
\item \code{if} and \code{else} is Scala
\item \code{if} is an expression that returns a value
\begin{itemize}
\item Like ``\code{cond ? a : b;}'' in C and Java
\end{itemize}
\item This is not a hardware multiplexer
\item Decides which module to generate
\item Could even read an XML file for the configuration
\end{itemize}
\end{frame}

\begin{frame}[fragile]{Conditional Component Generation}
\begin{chisel}
val icache =
  if (TYPE == METHOD)
    Module(new MCache())
  else if (TYPE == LINE)
    Module(new ICache())
  else
    ChiselError.error("Unsupported Type")
\end{chisel}
\begin{itemize}
\item Use Scala if/else for conditional component types
\item Code example from Patmos
\item We parse an XML file for the configuration
\end{itemize}
\end{frame}

\begin{frame}[fragile]{Combinational (Truth) Table Generation}
\begin{chisel}
val arr = new Array[Bits](length)
for (i <- 0 until length) {
  arr(i) = ...
}
val rom = Vec[Bits](arr)
\end{chisel}
\begin{itemize}
\item Generate a table in a Scala array
\item Use that array as input for a Chisel \code{Vec}
\item Generates a logic table at hardware construction time
\end{itemize}
\end{frame}

\begin{frame}[fragile]{Logic Generation}
\begin{itemize}
\item Read a file into a table
\begin{itemize}
\item E.g., to read in ROM content for a processor
\end{itemize}
\item Generate a truth table algorithmically
\begin{itemize}
\item E.g., generate binary to BCD translation
\end{itemize}
\item Use the full power of Scala
\end{itemize}
\begin{chisel}
val byteArray = Files.readAllBytes(Paths.get(fileName))
val arr = new Array[Bits](byteArray.length)
for (i <- 0 until byteArray.length) {
  arr(i) = Bits(byteArray(i), 8)
}
val rom = Vec[Bits](arr)
\end{chisel}
\end{frame}

\begin{frame}[fragile]{Ideas for Runtime Table Generation}
\begin{itemize}
\item Assembler in Scala/Java generates the boot ROM
\item Table with a \code{sin} function
\item Binary to BCD conversion
\item Schedule table for a TDM based network-on-chip
\item 
\item More ideas?
\end{itemize}
\end{frame}

\begin{frame}[fragile]{Memory}
\begin{chisel}
val mem = Mem(Bits(width = 8), size)

// write
when(wrEna) {
  mem(wrAddr) := wrData
}

// read
val rdAddrReg = Reg(next = rdAddr)
rdData := mem(rdAddrReg)
\end{chisel}
\begin{itemize}
\item Write is synchronous
\item Read can be asynchronous or synchronous
\item But there are no asynchronous memories in an FPGA
\end{itemize}
\end{frame}

\begin{frame}[fragile]{Factory Methods}
\begin{itemize}
\item Simpler component creation and use
\item Usage similar to built in components, such as \code{Mux}
\end{itemize}
\begin{chisel}
val myAdder = Adder(x, y)
\end{chisel}
\begin{itemize}
\item A little bit more work on component side
\item Define an \code{apply} method on the companion object that returns the component
\end{itemize}
\begin{chisel}
object Adder {
  def apply(a: UInt, b: UInt) = {
    val adder = Module(new Adder)
    adder.io.a := a
    adder.io.b := b
    adder.io.result
  }
}
\end{chisel}
\end{frame}

\begin{frame}[fragile]{Links to the Examples}
\begin{itemize}
\item Example code
\end{itemize}
\begin{chisel}
https://github.com/schoeberl/chisel-examples.git
\end{chisel}
\begin{itemize}
\item Slides
\end{itemize}
\begin{chisel}
https://github.com/schoeberl/chisel-book.git
\end{chisel}
\end{frame}

\begin{frame}[fragile]{Feedback}
\begin{itemize}
\item Plans on exploring Chisel?
\item What went well?
\item What was not so good?
\item How can this Chisel course be improved?
\item
\item I can also do a more hands on version of this course (with prepared laptops and FPGA boards)
\item
\item Would be happy to receive an email: \code{masca@dtu.dk}
\end{itemize}
\end{frame}

\end{document}